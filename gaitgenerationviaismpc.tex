\chapter{Gait generation via IS-MPC}
\label{ch:ISMPC}
In the last chapter, we have given an overview on humanoid locomotion dynamics,
focusing in particular on the model of the Linear Inverted Pendulum (LIP). We have
described its dynamics, we have seen that it is unstable, and we have given a
condition for contact equilibrium on non-coplanar surfaces for the humanoid.
In this chapter, we exploit the previously
presented analysis to develop a gait generation scheme for humanoid locomotion
in a \textit{world of stairs}, an uneven terrain where all contact surfaces 
are piecewise horizontal.

The approach adopted in this manuscript is based on IS-MPC \cite{Scianca2020TRO},
a model predictive control scheme which, given as input a \textit{footstep plan}
(which defines the desired motion of the robot at high-level specifying desired
footstep positions, orientations, single and double support durations), generates
a stable center of mass (CoM) trajectory, which can be tracked by a whole-body
controller. In IS-MPC, where the model is the one of the LIP, stability is
guaranteed via a stability constraint, which
bounds the displacement between the CoM and the zero-tilting moment point (ZMP).
Moreover, dynamic balance is enforced via constraints on the ZMP
position. In particular, since the polyhedral cone of eq.
\eqref{eq:ZMP-polyhedral-cone} would result in a nonlinear constraint, it is
conservatively approximated with a convex region. Because both constraints are 
linear, IS-MPC can be formulated as a Quadratic Programming (QP) program, and 
solved efficiently at each control iteration.

This chapter presents IS-MPC in its entirety, describing the prediction model (a LIP with a 
dynamic extension on the ZMP), the stability and the balance constraint, and the
definition of the QP problem. In the end, the condition of feasibility of IS-MPC
optimization problem is discussed.

\section{Prediction model}

Control of the ZMP is achieved using a dynamic model relating the position of the latter to the position and acceleration of the CoM. This dynamic model can be derived by balancing moments on the humanoid as a whole, and assuming that the rate of change of angular momentum around the CoM can be neglected, leading to the situtation shown in Fig. \ref{fig:LIPM-robot}, where the Center of Pressure (CoP) is located on the corresponding patch, while the ZMP can be anywhere along the line joining the CoP and the CoM \cite{Caron2017DynamicWalkingOverRoughTerrains}.

Denote the CoM as $\bfp_c = (x_c \; y_c \; z_c)^T$ and the ZMP as $\bfp_z = (x_z \; y_z \; z_z)^T$.

The choice of restricting the available trajectories to those resulting in a constant $\eta$ allows to make the prediction model linear. If this restriction is removed, the model is referred to as Variable-Height Inverted Pendulum (VH-IP), which can be treated either as nonlinear or time-varying. This can allow for more general motions to be generated, e.g., running~\cite{Smaldone2022Running}, at the cost of a slightly more complex architecture. For the present case, where only walking is considered, the simpler model is preferred.

In order to obtain smoother trajectories, model (\ref{eq:WoS:3dmodel}) is dynamically extended to have the derivative of the ZMP $\dot \bfp_z$ as the input.
The gait generation scheme works over discrete time-steps of duration $\delta$, over which the input $\dot \bfp_z$ is assumed to be constant, i.e.,
\begin{equation*}
    \dot \bfp_z(t) = \dot \bfp_z^k,
\end{equation*}
for $t\in[t_k, t_{k+1})$. This prediction model is used to forecast the evolution of the system over a receding horizon window called the \emph{control horizon}, spanning a time $T_c=C\delta$. The number of steps that are contained, either fully or partially, within this control horizon is denoted as $F$.

\section{Stability constraint}
Model~(\ref{eq:WoS:3dmodel}) has a positive eigenvalue $\eta$, reflecting the intrinsic instability of the humanoid dynamics. Given this instability, it is not sufficient to generate a gait such that the ZMP is inside the support region, because the associated CoM trajectory might be divergent, making the motion unrealizable by the humanoid.
The role of the stability constraint is to enforce a condition on the unstable component of the dynamics in order to guarantee that the CoM trajectory does not diverge with respect to the ZMP.

Despite the instability, the evolution of the system is bounded if the following \emph{stability condition} is satisfied
\begin{equation}
\label{eq:WoS:stabilitycondition}
\bfp_u^k = \eta\int_{t_k}^{\infty}e^{-\eta(\tau - t_k)}\bfp_z(\tau) d\tau - \frac{\bfg}{\eta^2},
\end{equation}
where the superscript in ${\bfp}_u^k$ indicates that the variable is sampled at time $t_k$.

Condition~(\ref{eq:WoS:stabilitycondition}) is non-causal as it requires knowledge of the future ZMP trajectory $\bfp_z$ up to infinity. In order to derive a causal implementation, we split the integral at $t_{k+C}$. Of the two separate integrals that result, the first, over $[t_k, t_{k+C})$, can be expressed in terms of the MPC decision variables. A value for the second integral, over $[t_{k+C}, \infty)$, can be obtained by conjecturing a ZMP trajectory using information coming from the footstep plan.
This conjectured trajectory is called \emph{anticipative tail} and is denoted with $\tilde \bfx_z$. 
In \cite{Scianca2020TRO}, the anticipative tail was used to prove recursive feasibility and stability of the MPC scheme.

The stability constraint is then written as
\begin{equation}\label{eq:WoS:stability_constraint}
\eta\int_{t_k}^{t_{k+C}}e^{-\eta(\tau - t_k)}\bfp_z d\tau = \bfp_u^k - \tilde \bfc^k + \frac{\bfg}{\eta^2}.
\end{equation}
where $\tilde \bfc^k$ is given by
\begin{equation}%\label{eq:WoS:stability_constraint}
\tilde\bfc^k = \eta\int_{t_{k+C}}^{\infty}e^{-\eta(\tau - t_k)}\tilde\bfp_z d\tau.
\end{equation}

Note that \cite{Scianca2020TRO} considers the footstep plan to be available over a receding window called the \emph{preview horizon}. Here there is no need to make such an assumption, as the footstep plan is provided in its entirety, and once the goal is reached the robot comes to a complete stop.

Enforcing constraint (\ref{eq:WoS:stability_constraint}) allows to bound the displacement between CoM and ZMP. In fact, the value of the bound is almost identical in most practical situation, especially in view of the fact that the preview horizon is unlimited because the plan is completely known.

\textbf{From Humanoids 2023 stability constraint:}
[...] The second step is to make the decision variables appear explicitly, i.e., the ZMP velocities $\dot{\bm{X}}_z$ over the control horizon, by computing the integral over a piecewise linear ZMP trajectory. The final form of the constraint can be found in \cite{Scianca2020TRO}. For the purpose of this analysis, we will use the compact expression
\begin{equation}\label{eq:FAPA:stability_constraint}
    \bfs^T \dot{\bm{X}}_z = b_x^k + x_u^k,
\end{equation}
where $\bfs\in\mathbb{R}^C$ and $b_x^k\in\mathbb{R}$ denote respectively a vector and a scalar whose explicit expressions can be recovered from the cited reference.


\section{ZMP constraint}
\subsection{Humanoids 2023}
\textbf{From Humanoids 2023 Stability constraint:} Relating the dynamics of the CoM to those of the ZMP is essential since the latter encodes information about the realizability of ground reaction forces, and thus provides a criterion for balance. 
% While on flat ground this criterion consists in keeping the ZMP within the support polygon of the robot, an extension for the 3D case is required. 
A common way to extend the basic 2D balance criterion consists in prescribing the 3D ZMP to be inside a 3D pyramid $\cal Z$ (see Fig.~\ref{fig:FAPA:balance3D}), having the base defined by the contact surfaces and the CoM vertex \cite{Sugihara2002ICRA, Cipriano2023RAS}.

The IS-MPC block receives the adapted subplan ${\cal P}^l$ and uses it to construct ZMP constraints. As described in the previous section, the criterion for balance is satisfied if the ZMP belongs to the pyramid $\cal Z$. However, enforcing this condition directly would lead to a nonlinear constraint in the MPC because the vertex of the pyramid is the CoM of the robot. Thus, we adopt a conservative approximation called the {\em moving constraint}.

The moving constraint requires for the ZMP to be at all times within a convex polyhedron of fixed shape, in our case a box of dimensions $d_x$, $d_y$ and $d_z$ centered in $\bfp_{\rm mc}=(x_{\rm mc}, y_{\rm mc}, z_{\rm mc})$, which we call the {\em moving box}. Along the prediction, the moving box can translate but not rotate, and its center moves in such a way that it is always fully contained within the 3D pyramid $\mathcal{Z}$ (see Fig. 5 in \cite{Zamparelli2018SYROCO}). The vector $\bm{X}_{\rm mc}^{k+1} = (x_{\rm mc}^{k+1}, \dots, x_{\rm mc}^{k+C})^T$ collects the $x$ coordinate of the center of the moving box in the control horizon.

Because of its constant orientation in the prediction, at each time we can choose the orientation of the axes to align with the orientation of the moving box (taken as the orientation of the current support foot) and obtain a ZMP constraint that is decoupled along the 3 axes. Focusing on the component along $x$, we can write it as
\begin{equation}\label{eq:FAPA:ZMP_constraints}
    \bm{X}_z^{\rm m,k+1} \le \bm{X}_z^{k+1} \le \bm{X}_z^{M,k+1},
\end{equation}
where $\bm{X}_z^{k+1} = (x_z^{k+1}, \dots, x_z^{k+C})^T$ is a vector of predicted ZMP positions, and $\bm{X}_z^{\rm m,k+1}$ and $\bm{X}_z^{\rm M,k+1}$ are the ZMP bounds along the prediction.
By defining
\begin{equation*}
    \bm{Z} =
    \begin{pmatrix}
        \delta & 0 & \cdots & 0 \\
        \delta & \delta & \cdots & 0 \\
        \vdots & \vdots & \ddots & \vdots \\
        \delta & \delta & \cdots & \delta
    \end{pmatrix},
    \qquad
    \bm{z} =
    \begin{pmatrix}
        1 \\ 1 \\ \vdots \\ 1
    \end{pmatrix},
\end{equation*}
$\bm{X}_z^{k+1}$ can be expressed as
\begin{equation}
    \label{eq:FAPA:zmp-model-matrix-form}
    \bm{X}_z^{k+1} = \bm{Z} \dot{\bm{X}}_z^{k} + \bm{z} x_z^k,
\end{equation}
where $\dot{\bm{X}}_z^{k} = (\dot x_z^{k}, \dots, \dot x_z^{k+C-1})^T$ is the vector of ZMP velocities, i.e., the decision variables. The ZMP bounds along the prediction can be expressed as
\begin{equation}
    \bm{X}_z^{\rm m,k+1} = \bm{X}_{\rm mc}^{k+1} - \bfz \frac{d_x}{2}, \quad 
    \bm{X}_z^{\rm M,k+1} = \bm{X}_{\rm mc}^{k+1} + \bfz \frac{d_x}{2}.
\label{eq:FAPA:zmp_constraint_displacement}
\end{equation}
The center of the moving box $\bfp_{\rm mc}$ must be expressed in terms of the subplan $\mathcal{P}^l$. First we define the {\em piecewise-linear sigmoid} function 
\begin{equation*}
\sigma (t,t_i,t_f)=\frac{1}{t_f-t_i} \left(\rho(t-t_i)-\rho(t-t_f)\right),
%\label{eq:FAPA:sigma}
\end{equation*}
where $\rho(t)=t\delta_{-1}(t)$ is the unit ramp. $\sigma (t,t_i,t_f)$ is 0 before $t_i$,  1 after $t_f$, and it transitions linearly in the interval $[t_i,t_f]$. This function is useful to represent the transition between consecutive footsteps.

$\bm{X}_{\rm mc}^{k+1}$ can be written as
\begin{equation}
\label{eq:FAPA:mapping}
\bm{X}_{\rm mc}^{k+1} = \bm{M} \bm{X}_f^l + \bm{m}x_f^l,
\end{equation}
where $\bm{X}_{f}^{l} = (x_{f}^{l}, \dots, x_{f}^{l+F})^T$ collects the footstep positions. $\bm{M}\in\mathbb{R}^{C\times F}$ is a mapping matrix whose elements $M_{ij}$ are defined as
\begin{equation}\begin{split}
M_{ij} &= \sigma(t_{k+i}, t_s^{l+j},t_s^{l+j}+T_{\rm ds}^{l+j})\\ &- \sigma(t_{k+i}, t_s^{l+j-1},t_s^{l+j-1}+T_{\rm ds}^{l+j-1}),
\end{split}\end{equation}
and $\bfm\in\mathbb{R}^{C}$ is a vector whose elements $m_i$ are given by
\begin{equation*}
m_{i} = 1 - \sigma(t_{k+i}, t_s^{l},t_s^{l}+T_{\rm ds}^{1}),
\end{equation*}
where $t_s^l$ is the starting time of the $l$-th step and
\begin{equation*}
    t^j_s = t^l_s + \sum^{l+j-1}_{\lambda=l} \left(T_{\rm ds}^\lambda + T_{\rm ss}^\lambda\right).
\end{equation*}

\subsection{RAS 2023}
As already noted, the humanoid is balanced as long as the ZMP is inside the pyramid $\cal Z$, which is a nonlinear condition due to the vertex of the pyramid being at the CoM (Fig. \ref{fig:FAPA:balance3d}). To preserve linearity we consider a smaller allowed region for the ZMP, consisting in a box of fixed size with changing center and orientation. We refer to this as the {\em moving box}\footnote{Approximating the pyramidal region $\cal Z$ with a box might seem overly conservative. However, we argue that the neglected portion of the pyramid region is not crucial here, because large displacements of the ZMP in the $z$ direction would only be required to generate large vertical accelerations, which are not necessary in the considered setting (walking in a world of stairs). Clearly, less conservative approximations can still be envisaged and used for generating more dynamic motions.}, and we will prove that it conservatively approximates the support region, meaning that it is always contained inside the pyramid $\cal Z$.

The center $\bfp_{\rm mc}$ of the moving box is taken to be consistent with the pose of footstep $\bff^j$ during the $j$-th single support phase. At each time $t$, the center of the moving box $\bfp_{\rm mc}$ is expressed as
\begin{equation}\label{eq:WoS: moving constraint}
\bfp_{\rm mc}(t) = \Bigg\{
\begin{array}{ll}
\bfp^j & t\in [t_s^j, t_s^j + T_{\rm ss}^j) \\
(1-\alpha^j(t))\bfp^j + \alpha^j(t)\bfp^{j+1} & t\in [t_s^j + T_{\rm ss}^j, t_s^{k+1})
\end{array}
\end{equation}
where $j=0,\dots,F$ is a index over the footsteps within the control horizon, and $\alpha^j(t) = (t-t_s^j-T_{\rm ss}^j)/T_{\rm ds}^j$ denotes the time elapsed since the start of the double support phase, expressed as a fraction of the duration of the double support phase itself.

The ZMP position constraint is expressed as
\begin{equation}\label{eq:WoS:ZMP_constraint}
-\tilde\bfd /2 \le \bfp_z^{k+i} - \bfp_{\rm mc}^{k+i} \le \tilde\bfd /2
\end{equation}
where $\bfp_z^{k+i}$ and $\bfp_{\rm mc}^{k+i}$ respectively denote the ZMP and the center of the moving box sampled at time $t_{k+i}$, $\tilde \bfd = (\tilde d_x, \tilde d_y, d_z)^T$
is a vector collecting the dimensions of the moving box along all three axes.

The size of the moving box $\tilde \bfd = (\tilde d_x, \tilde d_y, d_z)^T$ is determined in such a way to always be contained inside the pyramid $\cal Z$. \textbf{TODO: either add Prop. 2 of RAS23 or eq. (12) of SYROCO18.}

\section{IS-MPC algorithm}
IS-MPC solves, at each time $t_k$, the following QP problem:
\begin{braced}
\begin{equation}
\begin{split}
\min_{\dot{\bm{X}}_\text{z}^k, \dot{\bm{Y}}_\text{z}^k, \dot{\bm{Z}}_\text{z}^k}
&\|\dot{\bm{X}}_z^k\|^2 + \|\dot{\bm{Y}}_z^k\|^2 + \|\dot{\bm{Z}}_z^k\|^2+\beta \|\bm{X}_z^{k+1} - \bfX_{\rm mc}^{k+1}\|^2 \\& + \beta \|\bm{Y}_z^{k+1} - \bfY_{\rm mc}^{k+1}\|^2 + \beta \|\bm{Z}_z^{k+1} - \bfZ_{\rm mc}^{k+1}\|^2 \\
\end{split}
\label{eq:ISMPC-algorithm}
\end{equation}
\hspace{0.25cm} subject to:
\begin{itemize}
\item ZMP constraints (\ref{eq:FAPA:ZMP_constraints})
\item stability constraints (\ref{eq:FAPA:stability_constraint})
\end{itemize}
\end{braced}

In the cost function, the first three terms act as regularization while the remaining attempt to bring the ZMP as close as possible to the center of the moving box, with a strength modulated by the weight $\beta$.

The first sample $\dot \bfp_z^k = (\dot x_z^k, \dot y_z^k, \dot z_z^k)$ of the optimal sequence is used to integrate the prediction model and the resulting CoM position $\bfp_c^{k+1}$ is sent to the kinematic controller together with a suitable swing foot trajectory that allows to reach the target footstep position at the proper time.

\section{Feasibility region}

The {\em feasibility region} is the region of the state space in which the
IS-MPC optimization problem \eqref{eq:ISMPC-algorithm} is feasible.

\begin{proposition}
\label{prop:feasibility}
IS-MPC is feasible at time $t_k$ if
\begin{align}
\label{eq:FAPA:mpc-feasibility-constraint}
\bm{s}^T \bm{Z}^{-1} (\bm{X}_z^{{\rm m}, k+1} \!-\! \bm{z} x_z^k)  \!&\le\! x_u^k \!+\! b_x^k \!\le\! \bm{s}^T \bm{Z}^{-1} (\bm{X}_z^{{\rm M}, k+1} \!-\! \bm{z} x_z^k),
\nonumber\\
\bm{s}^T \bm{Z}^{-1} (\bm{Y}_z^{{\rm m}, k+1} \!-\! \bm{z} y_z^k)  \!&\le\! y_u^k \!+\! b_y^k \!\le\! \bm{s}^T \bm{Z}^{-1} (\bm{Y}_z^{{\rm M}, k+1} \!-\! \bm{z} y_z^k),
\nonumber\\
\bm{s}^T \bm{Z}^{-1} (\bm{Z}_z^{{\rm m}, k+1} \!-\! \bm{z} z_z^k)  \!&\le\! z_u^k \!+\! b_z^k \!\le\! \bm{s}^T \bm{Z}^{-1} (\bm{Z}_z^{{\rm M}, k+1} \!-\! \bm{z} z_z^k).
\end{align}
\end{proposition}
{\em Proof}.
We focus the proof on the inequalities for the $x$ component, as the logic, for the other components is identical. The bounds of the feasibility region along $x$ are given by
\begin{align*}
x_u^{k,b1} &= \bm{s}^T \bm{Z}^{-1} (\bm{X}_z^{{\rm m}, k+1} - \bm{z} x_z^k) - b_x^k, \\
x_u^{k,b2} &= \bm{s}^T \bm{Z}^{-1} (\bm{X}_z^{{\rm M}, k+1} - \bm{z} x_z^k) - b_x^k.
\end{align*}
Then, if $x_u^k$ is inside the feasibility region, it is possible to express it as a convex combination of the two bounds, i.e.,
\begin{equation}\label{eq:FAPA:xu_convex}
x_u^k = \alpha x_u^{k,b1} + (1-\alpha)x_u^{k,b2}, \alpha \in [0, 1].
\end{equation}

Consider the following ZMP velocity trajectory:
\begin{equation}\label{eq:FAPA:candidate_trajectory}
\dot{\bm{X}}_z^k = \alpha \bm{Z}^{-1}(\bm{X}_z^{{\rm m}, k+1} - \bm{z} x_z^k) + (1-\alpha)\bm{Z}^{-1}(\bm{X}_z^{{\rm M}, k+1} - \bm{z} x_z^k).
\end{equation}
We will show that this particular trajectory satisfies both the stability constraint and the ZMP constraints. As for the stability constraint, multiply both sides of (\ref{eq:FAPA:candidate_trajectory}) by $\bm{s}^T$
% \begin{equation}
% \bm{s}^T\bm{\dot X}_z^k = \bm{s}^T(\alpha \bm{Z}^{-1}(\bm{X}_z^{{\rm m}, k+1} - \bm{z} x_z^k) + (1-\alpha)\bm{Z}^{-1}(\bm{X}_z^{{\rm M}, k+1} - \bm{z} x_z^k)),
% \end{equation}
and plug in the definitions of $x_u^{k,b1}$ and $x_u^{k,b2}$ to obtain
\begin{equation*}
\bm{s}^T\dot{\bm X}_z^k = (\alpha (x_u^{k,b1} + b^k_x)) + (1-\alpha)(x_u^{k,b2} + b^k_x)).
\end{equation*}
Using (\ref{eq:FAPA:xu_convex}), this is equivalent to the stability constraint (\ref{eq:FAPA:stability_constraint}).

To prove satisfaction of the ZMP constraint. Left-multiplying \eqref{eq:FAPA:zmp-model-matrix-form} by $\bm{Z}$, the chosen ZMP velocity trajectory can be rewritten as
\begin{equation*}
\bm{X}_z^k - \bm{z} x_z^k = \alpha (\bm{X}_z^{{\rm m}, k+1} - \bm{z} x_z^k) + (1-\alpha)(\bm{X}_z^{{\rm M}, k+1} - \bm{z} x_z^k),
\end{equation*}
which simplifies to
$\bm{X}_z^k = \alpha \bm{X}_z^{{\rm m}, k+1} + (1-\alpha)\bm{X}_z^{{\rm M}, k+1}$,
and therefore the ZMP constraint (\ref{eq:FAPA:ZMP_constraints}) is satisfied. \hfill\bull
