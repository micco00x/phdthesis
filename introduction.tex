\chapter{Introduction}
Robots are shaping our world, progressively becoming an integral part of our 
society. From manipulators in production lines, autonomous vehicles in 
space, surgical systems in hospitals to vacuum cleaners in our homes, we can 
literally say that robots are everywhere. Nevertheless, while the adoption of
autonomous systems has skyrocketed in the past few decades, and research has
made great strides, the technology of today
is not yet ready for the adoption of more complex robots, such as humanoids
and mobile manipulators.

Humanoids robots, thanks to their anthropomorphic structure, are in principle 
expected of interacting with our environment, in the same way we humans do.
One of the biggest challenges in the research of humanoids is that of 
locomotion in 3D environments, for which complex motions may be required.
Indeed, in order for humanoid robots to properly perform locomotion tasks, 
steeping over or onto obstacles, climbing and descending stairs, and 
overcoming gaps must be taken into account. 

Mobile manipulators, on the other hand, are not designed to move in such 
complex environments, but may possess the same manipulation capabilities if 
equipped with proper arms. The lower body of mobile manipulators consists
of a mobile base, which differs depending on the tasks the robot is
supposed to solve. Among the possible choices, employing a mobile base equipped 
with steerable wheels allows the robot to perform complex motions (because 
of the omnidirectionality of the base) without sacrificing its robustness
(because of the presence of classical wheels).

While, in literature, there exists a large number of methodologies for
generating and controlling the motion of mobile robots, the problems 
of locomotion of humanoids in 3D environment, and locomotion of robots 
equipped with steerable wheels (steerable wheeled mobile robots), have not been 
entirely solved yet. The reason lies in the complexity of the problem itself.
In order to address locomotion, one should take into account a multitude of 
subproblems, such as perception (how the robot sees its sorroudings,
and how to build a representation of the environment), motion 
planning (how to use such representation and efficiently plan how the robot 
is going to move), and control (how to actually perform the motion, possibly 
in a robust way, taking into account disturbances and external perturbations).

The focus of this manuscript is that of studying the aforementioned problems,
and proposing new techniques which exploit modern planning and control
algorithms.

\section{Contribution}
\subsubsection{Humanoid motion generation in a world of stairs}
The first contribution of this thesis is a framework for the generation of 
motion of humanoid robots in a world of stairs, an uneven terrain 
where all contact surfaces are piecewise-horizontal. The problem is first 
addressed by assuming complete knowledge of the environment, and then 
extended to the case in which the environment is unknown. In order to 
develop such scheme, localization, mapping, motion planning 
and control modules must efficiently work together. We propose to solve this 
problem by implementing a footstep planner based on RRT*, which 
plans a sequence of footsteps that brings the robot from its current 
configuration to a desired goal region. Our footstep planner is 
capable of replanning the sequence of footsteps during the execution of the 
motion, improving the footstep plan, and taking into account changes 
in the map (we map the environment in real-time using an elevation mapping 
module) due to dynamic obstacles. Moreover, our scheme
uses a model predictive control (MPC) algorithm that guarantees that the humanoid is
dynamically balanced at all times. This work led to a journal paper, which 
has been published in \textit{Robotics and Autonomous Systems} 
\cite{Cipriano2023RAS}.

\subsubsection{Feasibility-aware plan adaptation in humanoid gait generation}
The second contribution of this thesis is a module for adapting footstep plans 
(positions, orientations, and timings) in such a way to increase robustness 
of humanoid robot locomotion in 3D environments.
Indeed, while the robot walks, it may be subject 
to external perturbations (such as pushes), which may make the robot lose its 
equilibrium. The problem is first addressed by considering the case in which 
adaptation does not allow to select a different contact surface with respect
to the one initially assigned to each footstep, and then extended to the case 
in which the contact surface may be reassigned, further increasing the external 
disturbances the robot can sustain. We propose to solve this problem by 
implementing the adaptation scheme as a nonlinear programming problem for the
first case, and as a mixed-integer nonlinear programming problem for the
second case. Our method, which is tightly integrated with our MPC scheme,
allows to perform adaptation without affecting the performance of the MPC
itself. This work led to a conference paper, which has been published 
in \textit{2023 IEEE-RAS International Conference on Humanoid
Robots}~\cite{Cipriano2023Humanoids}.

\subsubsection{Nonlinear model predictive control for steerable WMRs}
The third contribution of this thesis is a framework for trajectory tracking 
for steerable wheeled mobile robots (steerable WMRs, or SWMRs).
The difficulty with this kind of platform,
as we will discuss more in detail later, is the presence of kinematic model 
singularities. These must be taken into account when developing a control 
scheme, possibly together with constraints on actuation, in order to generate
feasible control actions. We propose to solve this problem by implementing a 
nonlinear MPC based on the real-time iteration scheme. Our MPC is supported by a 
finite state machine, and an auxiliary trajectory generation scheme based 
on dynamic feedback linearization, which guarantee kinematic singularities 
are never encountered, and actutation constraints are always satisfied.
At the time of writing of this manuscript, this work is about to be 
submitted to \textit{IEEE Robotics and Automation Letters}.

\section{Outline}
The manuscript is organized as follows.
\begin{itemize}
    \item Chapter \ref{ch:humanoids-and-swmrs} presents an historical
        overview on humanoid robots and steerable wheeled mobile robots,
        discussing the role of robotics in our society and in research.
    \item Chapter \ref{ch:literature-review} reviews the literature on the topics 
        of gait generation, footstep planning, sensor-based locomotion, and footstep 
        and timing adaptation for humanoid robots, and motion control for steerable 
        WMRs.
\end{itemize}
Part I: Motion generation for humanoid robots
\begin{itemize}
    \item Chapter \ref{ch:humanoid-locomotion-dynamics} introduces the dynamics
        of humanoid robots, focusing in particular on the Linear
        Inverted Pendulum (LIP), and the condition of equilibrium for humanoids
        in the world of stairs.
    \item Chapter \ref{ch:ISMPC} presents Intrinsically-Stable MPC (IS-MPC), the gait 
        generation scheme used in the subsequent chapters,
        which takes into account the dynamics of the LIP, and guarantees the
        stability of the system via a stability constraint.
    \item Chapter \ref{ch:humanoid-motion-generation-WoS} presents a framework 
        for motion generation in a world of stairs, which is composed of a 
        footstep planner based on RRT*, a gait generation scheme based on IS-MPC,
        a module to map the environment as an elevation map, and a localization 
        system based on SLAM.
    \item Chapter \ref{ch:FAPA} presents Feasibility-Aware Plan Adaptation
        (FAPA), a module for footstep plan adaptation in 3D environments
        which works alongside IS-MPC.
\end{itemize}
Part II: Motion control for steerable WMRs
\begin{itemize}
    \item Chapter \ref{ch:nmpc-rti} briefly reviews the Nonlinear MPC scheme 
        based on real-time iteration (RTI), introducing the algorithms 
        and the notation used in the subsequent chapter.
    \item Chapter \ref{ch:nmpc-swmr} presents a framework for trajectory 
        tracking for steerable WMRs, which makes use of a Nonlinear MPC
        based on RTI.
    \item Chapter \ref{ch:conclusions} concludes the thesis, summarizing the 
        contributions and discussing future works.
\end{itemize}