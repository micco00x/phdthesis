\chapter{Introduction}
Robots are shaping our world, progressively becoming an integral part of our 
society. From manipulators in production lines, autonomous vehicles in 
space, surgical systems in hospitals to vacuum cleaners in our homes, we can 
literally say that robots are everywhere. Nevertheless, while the adoption of
autonomous systems has skyrocketed in the past few decades, and research has
made great strides, the technology of today
is not yet ready for the adoption of more complex robots, such as humanoids
and mobile manipulators.

Humanoids robots, thanks to their anthropomorphic structure, are in principle 
capable of interacting with our environment, in the same way we humans do.
One of the biggest challenges in the research of humanoids is that of 
locomotion in 3D environments, for which complex motions may be required.
Indeed, in order for humanoid robots to properly perform locomotion tasks, 
steeping over or onto obstacles, climbing and descending stairs, and 
overcoming gaps must be taken into account. 

Mobile manipulators, on the other hand, are not designed to move in such 
complex environments, but may possess the same manipulation capabilities if 
equipped with proper arms. The lower body of mobile manipulators consists
of a mobile base, which differs depending on the tasks the robot is
supposed to solve. Among the possible choices, employing a mobile base equipped 
with steerable wheels allows the robot to perform complex motions (because 
of the omnidirectionality of the base) without sacrificing its robustness
(because of the presence of classical wheels).

While, in literature, there exists a large number of methodologies for
generating and controlling the motion of mobile robots, the problems 
of locomotion of humanoids in 3D environment, and locomotion of robots 
equipped with steerable wheels (steerable wheeled mobile robots), has not been 
entirely solved yet. TODO: say the main reasons [...].

TODO: The focus of this manuscript is [...].

\section{Contribution}
Thesis contribution and list of publications.

\section{Outline}
Outline of the thesis.
