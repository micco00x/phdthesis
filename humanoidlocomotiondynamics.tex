\chapter{Dynamics of humanoid locomotion}
Intro. Robot walks by exchanging forces with the environment. Dynamic balance, contacts. Simplified models and contact equilibrium.

\section{Lagrangian dynamics}
\label{sec:lagrangian-dynamics}
Define configuration of the robot:
\begin{equation*}
    \ddot{\bm{q}} =
    \begin{bmatrix}
        \bm{q}_b \\ \bm{q}_j
    \end{bmatrix} \in \mathrm{SE}(3) \times \mathrm{SO}(2)^{n_j}
\end{equation*}

Equations of motion
\begin{equation}
    \begin{bmatrix}
        \bm{M}_u \\ \bm{M}_a
    \end{bmatrix} \ddot{\bm{q}} +
    \begin{bmatrix}
        \bm{c}_u(\bm{q}, \dot{\bm{q}}) \\
        \bm{c}_a(\bm{q}, \dot{\bm{q}}) \\
    \end{bmatrix} =
    \begin{bmatrix}
        \bm{0} \\ \bm{\tau}
    \end{bmatrix} +
    \sum_{k=1}^{K}
    \begin{bmatrix}
        \bm{J}_{k, u}^T \\ \bm{J}_{k, a}^T
    \end{bmatrix}
    \bm{f}_k
    \label{eq:equation-of-motion-humanoids}
\end{equation}

Contact forces inside the friction cone to avoid slipping

A contact force $\bm{f}_k$ is \textit{feasible} if it lies in the friction cone $\mathcal{C}_k$ directed by the contact normal $\bm{n}_k$:
\begin{equation*}
    \| \bm{f}_k - (\bm{f}_k \cdot \bm{n}_k) \bm{n}_k \|_2 \le \mu_k (\bm{f}_k \cdot \bm{n}_k)
\end{equation*}
with $\mu_k$ static friction coefficient.

In the following we will assume that there always exists joint torques
$\bm{\tau}$ that realize the actuated part of eq.
\eqref{eq:equation-of-motion-humanoids}.

\section{Centroidal dynamics}
The above hypothesis allows as to focus on the unactuated part of the equation
\eqref{eq:equation-of-motion-humanoids}, and define the
\textit{centroidal dynamics} \cite{Orin2013CentroidalDynamics} of the humanoid:
\begin{equation}
    \label{eq:centroidal-dynamics}
    \begin{bmatrix}
        m \ddot{\bm{p}}_C \\ \dot{\bm{L}}_C
    \end{bmatrix} =
    \begin{bmatrix}
        m \bm{g} \\ \bm{0}
    \end{bmatrix} +
    \sum_{k=1}^K
    \begin{bmatrix}
        \bm{f}_k \\ (\bm{p}_C - \bm{p}_k) \times \bm{f}_k
    \end{bmatrix},
\end{equation}
where $m$ is the total mass of the robot, $\bm{p}_C$ is the position of its
center of mass (CoM),
$\bm{g} = (0 \; 0 \; -g)^T$ is the gravity vector, $\bm{f}_k$ is the contact force
applied at a point with coordinates $\bm{p}_k$ over a contact surface with normal
$\bm{n}_k$, $K$ is the total number of contacts, and $\bm{L}_c$ is the angular
momentum of the robot taken at the CoM.

Let us define the \textit{gravito-inertial wrench} taken at point $O$ as
\begin{equation}
    \label{eq:gravito-intertial-wrench}
    \bm{w}_O^{\rm gi}
    =
    \begin{bmatrix}
        \bm{f}^{\rm gi}\\
        \bm{\tau}_O^{\rm gi}
    \end{bmatrix}
    =
    \begin{bmatrix}
        m \bm{g} - m \bm{\ddot{p}}_C \\
        (\bm{p}_C - \bm{p}_O) \times (m \bm{g} - m \bm{\ddot{p}}_C) - \bm{\dot{L}}_C
    \end{bmatrix}.
\end{equation}

Similarly, the \textit{contact wrench} $\bm{w}_O$ can be defined as
\begin{equation}
    \label{eq:contact-wrench}
    \bm{w}_O^{\rm c}
    =
    \begin{bmatrix}
        \bm{f}^{\rm c} \\
        \bm{\tau}_O^{\rm c}
    \end{bmatrix}
    =
    \sum_{k=1}^K
    \begin{bmatrix}
        \bm{f}_k\\
        (\bm{p}_k - \bm{p}_O) \times \bm{f}_k
    \end{bmatrix}.
\end{equation}

Note that the centroidal dynamics \eqref{eq:centroidal-dynamics} can be
rewritten as a sum of the two above wrenches
\begin{equation}
    \bm{w}_O^{\rm gi} + \bm{w}_O^c = 0.
\end{equation}

\section{Zero-tilting moment point}
\label{sec:zero-tilting-moment-point}
Consider the gravito-inertial wrench defined in \ref{eq:gravito-intertial-wrench}. Zero-tilting moment points (ZMPs) are points $Z$ where the moment of the contact wrench aligns with the normal $\bm{n}$ of the contact surface \cite{SardainBessonnet2004}, i.e.,
\begin{equation}
    \label{eq:zmp-non-tilting-condition}
    \bm{\tau}_Z^{\rm gi} \times \bm{n} = \bm{0},
\end{equation}
which, using Varignon formula\footnote{A screw $\bm{w}_O =
(\bm{f},\bm{\tau}_O)$ represents the generalized force acting on a rigid body
\cite{Featherstone2007RigidBodyDynamicsAlgorithms},
and it is composed by a linear force $\bm{f}$ passing through $O$, together with the
total moment $\bm{\tau}_O$ about $O$. That total moment around any other point
$A$ can be computed using Varignon formula as $\bm{\tau}_A=\bm{\tau}_O+\bm{f}\times(\bm{p}_A-\bm{p}_O)$.}, can be rewritten as
\begin{equation*}
    \left(\bm{\tau}_O^{\rm gi} + (\bm{p}_O - \bm{p}_Z) \times \bm{f}^{\rm gi}\right) \times \bm{n} = \bm{0},
\end{equation*}
which, developing the triple cross product\footnote{The triple cross product
between three vectors $\bm{a}, \bm{b}, \bm{c} \in \mathrm{R}^n$ is defined as
the cross product of the vector $\bm{a}$ with the cross product of the other
two: $\bm{a}\times(\bm{b}\times\bm{c})=(\bm{a}\cdot\bm{c})\bm{b}-
(\bm{a}\cdot\bm{b})\bm{c}$. Note that, since the cross product is anticommutative,
the following holds:
($\bm{a}\times\bm{b})\times\bm{c}=-(\bm{c}\cdot\bm{b})\bm{a}+
(\bm{c}\cdot\bm{a})\bm{b}$.}, the above equation becomes
\begin{equation}
    \bm{\tau}_O^{\rm gi} \times \bm{n} - (\bm{n} \cdot \bm{f}^{\rm gi}) (\bm{p}_O - \bm{p}_Z) - \left(\bm{n} \cdot (\bm{p}_O - \bm{p}_Z)\right) \bm{f}^{\rm gi} = \bm{0}.
\end{equation}

Assuming that a point $Z$ lies on a plane with normal $\bm{n}$ intersecting the point $O$, i.e. $Z \in \Pi(O, n)$, the term $\bm{n} \cdot (\bm{p}_O - \bm{p}_Z) = 0$, and the above equation can be easily rewritten as
\begin{equation}
    \bm{p}_Z = \bm{p}_O + \frac{\bm{n} \times \bm{\tau}_O^{\rm gi}}{\bm{n} \cdot \bm{f}^{\rm gi}},
\end{equation}
finally defining the ZMP $Z$. Note that, more in general, there exists an infinity of ZMPs which lie on the non-central axis defined by \eqref{eq:zmp-non-tilting-condition}. For more details, please refer to \cite{SardainBessonnet2004}.

\subsection{Relationship between CoM, ZMP and angular momentum}
Consider the non-tilting condition of Eq. \eqref{eq:zmp-non-tilting-condition}. Using Varignon formula $\bm{\tau}_Z^{\rm gi} = \bm{\tau}_C^{\rm gi} + \bm{f}^{\rm gi} \times (\bm{p}_Z - \bm{p}_C)$, we have that
\begin{equation}
    \left(\bm{\tau}_C^{\rm gi} + \bm{f}^{\rm gi} \times (\bm{p}_Z - \bm{p}_C)\right) \times \bm{n} = \bm{0},
\end{equation}
which, computing the triple product, becomes
\begin{equation}
    \bm{\tau}_C^{\rm gi} \times \bm{n} - \left(\bm{n} \cdot (\bm{p}_Z - \bm{p}_C)\right) \bm{g}^{\rm gi} + (\bm{n} \cdot \bm{f}^{\rm gi}) (\bm{p}_Z - \bm{p}_C) = \bm{0}.
\end{equation}

Applying the definition of \textit{gravito-inertial wrench} of eq. \eqref{eq:gravito-intertial-wrench} and rearranging the terms, it is simple to prove \cite{Caron2017TRO} the following relationship between the CoM acceleration, the ZMP position and the angular momentum:
\begin{equation}
    \label{eq:relationship-com-zmp-angular-momentum}
    \ddot{\bm{p}}_C = \bm{g} + \frac{\bm{n} \cdot (\bm{\ddot{p}}_C - \bm{g})}{\bm{n} \cdot (\bm{p}_C - \bm{p}_Z)} (\bm{p}_C - \bm{p}_Z) + \frac{\bm{n} \times \bm{\dot{L}}_C}{m \left(\bm{n} \cdot (\bm{p}_C - \bm{p}_Z)\right)}.
\end{equation}
\section{Inverted pendulum models}
\subsection{Variable-Height Inverted Pendulum}
Consider again the centroidal dynamics of eq. \eqref{eq:centroidal-dynamics}
and assume that the rate of change of angular momentum is negligible (i.e.,
$\dot{\bm{L}}_C=\bm{0}$). We define the dynamics of the \textit{Variable-Height
Inverted Pendulum} (VH-IP) \cite{Koolen2016VHIP} as
\begin{equation}
    \label{eq:VH-IP}
    m \ddot{\bm{p}}_C = m \bm{g} + \bm{f}^{\mathrm{c}}.
\end{equation}

Note that, because $\dot{\bm{L}}_C=\bm{0}$, the contact force
$\bm{f}^{\mathrm{c}}$ can be parametrized \cite{Caron2020ICRA} as
\begin{equation}
    \label{eq:contact-force-with-LG-0}
    \bm{f}^{\mathrm{c}} = m \lambda(t) (\bm{p}_C - \bm{p}_Z),
\end{equation}
with $\lambda(t)$ natural frequency of the VH-IP (where we explicitely
denote the time dependency of lambda to highlight the nonlinearity in the
dynamics). Note that $\lambda(t) > 0$ because
of unilaterality of contact (i.e., it is not possible to pull on the ground).
The dynamics of the VH-IP can be rewritten as
\begin{equation}
    \label{eq:VH-IP-LG-0}
    \ddot{\bm{p}}_C = \lambda(t) (\bm{p}_C - \bm{p}_Z) + \bm{g}
\end{equation}
by plugging the contact force \eqref{eq:contact-force-with-LG-0} into
eq. \eqref{eq:VH-IP}.

%During locomotion, since the the $\bm{f}^{\mathrm{c}}$ are such that its
%component along the $z$ axis is positive (i.e., $f_z^{\mathrm{c}}=
%\bm{e}_z^T \bm{f}^{\mathrm{c}} > 0$ with $\bm{e}_z = (0 \; 0 \; 1)^T$), from eq.
%\eqref{eq:contact-force-with-LG-0}, we have that
%\begin{equation*}
%    f_z^{\mathrm{c}} = m \lambda(t) (z_C - z_Z) > 0,
%\end{equation*}
%which implies that $z_C > z_Z$.

\subsection{Linear Inverted Pendulum}
The dynamics of the VH-IP, as already mentioned, is nonlinear due to the
variable frequency $\lambda(t)$. In this
section, we derive the dynamics of the \textit{Linear Inverted Pendulum} (LIP)
\cite{Kajita2016IntroductiontoHumanoidRobotics}. To do so,
we constrain the vertical motion of the CoM \cite{Zamparelli2018SYROCO} so that
\begin{equation}
    \lambda(t) = \frac{\ddot{z}_C + g}{z_C - z_Z} =
    \frac{\bm{n} \cdot (\ddot{\bm{p}}_C - \bm{g})}{\bm{n} \cdot (\bm{p}_C - \bm{p}_Z)} =
    \eta^2,
\end{equation}
with $\eta$ an arbitrary constant. In this way,
the dynamics \eqref{eq:VH-IP-LG-0} becomes
\begin{equation}
    \label{eq:LIPM}
    \ddot{\bm{p}}_C = \eta^2 (\bm{p}_C - \bm{p}_Z) + \bm{g},
\end{equation}
which is linear. The dynamics of the LIP is obtained by choosing $\eta^2=g/h$,
with $h=z_C-z_Z$.

When walking on flat floor (a single contact surface with normal
$\bm{e}_z=(0 \; 0\; 1)^T$),
a common choice is to constrain the ZMP to lie on
a plane which coincides with the ground (i.e., without loss of generality
$z_Z=0$). As a consequence, the CoM is constrained to lie on a parallel plane
displaced by $h$ from the ZMP plane (i.e., $z_C=h$), and
the LIP dynamics further simplifies to the following dynamics:
\begin{align*}
    \ddot{x}_C &= \eta^2 (x_C - x_Z) \\
    \ddot{y}_C &= \eta^2 (y_C - y_Z).
\end{align*}
Note that, in this particular case, the ZMP coincides with the center of
pressure (CoP), which is the point of application of the contact force
$\bm{f}^{\mathrm{c}}$ on the ground \cite{SardainBessonnet2004}.

\section{Contact equilibrium}
Todo (describe pyramid in SYROCO18, which is a particular case of the 3D LIPM).

We are interested in defining the condition of equilibrium of a humanoid robot.
Let us focus on the special case of 3D LIPM dynamics with all contact forces
directed towards the CoM (which is a particular case of the 3D LIPM).
In this particular case, the contact forces can be rewritten as
\begin{equation}
    \label{eq:contact-force-towards-com}
    \bm{f}_k = \frac{\bm{p}_C - \bm{p}_k}{\| \bm{p}_C - \bm{p}_k \|_2} f_k,
\end{equation}
with $f_k$ the norm of $\bm{f}_k$. By plugging eq.
\eqref{eq:contact-force-towards-com} into eq.
\eqref{eq:centroidal-dynamics}, we obtain
\begin{equation}
    m \ddot{\bm{p}}_C = m \bm{g} + \sum_{k=1}^K \frac{\bm{p}_C - \bm{p}_k}{\| \bm{p}_C - \bm{p}_k \|_2} f_k.
\end{equation}
Moreover, since we are assuming to be in 3D LIPM, we can use \eqref{eq:LIPM} to
obtain (after rearranging the terms), the following:
\begin{equation*}
    \bm{p}_Z = \bm{p}_C - \sum_{k=1}^K \frac{\bm{p}_C - \bm{p}_k}{\| \bm{p}_C - \bm{p}_k \|_2} \frac{f_k}{m \eta^2}.
\end{equation*}

Because of the assumption of sufficient joint torque actuation made in Sec.
\ref{sec:lagrangian-dynamics}, in principle, we could modulate $f_k$ to position
the ZMP within the area defined by the following set:
\begin{equation}
    \mathcal{S}_Z = \left\{ \bm{p}_Z \,\middle\vert\, \bm{p}_Z = \bm{p}_C + \sum_{k=1}^K \gamma_k (\bm{p}_k - \bm{p}_C), \; \gamma_k > 0  \right\},
\end{equation}
which is a polyhedral cone with apex $\bm{p}_C$,
is known in literature as the support region, and  contains all feasible
positions of the ZMP.
