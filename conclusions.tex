\chapter{Conclusions}
This thesis addressed the problem of generation and control of motion for 
humanoids and steerable wheeled 
mobile robots. In particular, in the first part of the thesis
we have studied the problem of motion generation for humanoids 
in a \textit{world of stairs}, a particular uneven terrain where the contact 
surfaces are piecewise-horizontal. After introducing their dynamics in Chapter
\ref{ch:humanoid-locomotion-dynamics}, and Intrinsically-Stable Model Predictive
Control (IS-MPC) in Chapter \ref{ch:ISMPC},
we have presented a sensor-based framework for locomotion which makes use of a
footstep planner 
based on RRT* in Chapter \ref{ch:humanoid-motion-generation-WoS}, and a 
feasibility-aware plan adaptation module based on mixed-integer nonlinear
optimization in Chapter \ref{ch:FAPA}. In the second part of the thesis,
after a brief overview of Nonlinear Model Predictive Control (NMPC) based on the
real-time iteration (RTI) scheme in Chapter \ref{ch:nmpc-rti}, we have 
introduced a framework based on RTI for the control of the motion of steerable wheeled 
mobile robots (SWMRs) in Chapter \ref{ch:nmpc-swmr}. In the following, we
summarize the main scientific contributions of this manuscript, and discuss 
possible future works.

\medskip

In Chapter \ref{ch:humanoid-motion-generation-WoS},
we addressed the problem of motion generation for a humanoid
robot that must reach a certain goal region walking in a \textit{world of stairs}.
We considered two versions of such problem: the off-line and on-line case. 
In the first, the geometry of the environment is completely known in advance,
while in the second, it is reconstructed by the robot itself during motion using
an on-board sensor. In both cases, available information about the environment
is maintained in the form of an elevation map. 

For the off-line case, we proposed an architecture working in two main stages:
footstep planning and gait generation.
First, a feasible footstep plan leading to the goal region is off-line computed
using a randomized algorithm that takes into account the plan quality specified
by a given optimality criterion.
Then, an intrinsically stable MPC-based scheme computes a CoM trajectory that
realizes the found footstep sequence, while guaranteeing dynamic balance and
boundedness of the CoM with respect to the ZMP at all time instants.

For the on-line case, we proposed an extension of the architecture for the
off-line case where footstep plans are computed in parallel to gait generation
and map building.
To this end, we presented a sensor-based version of the footstep planner that
uses the knowledge about the environment incrementally acquired by the robot
during motion.

We validated the proposed architectures by providing simulation results
obtained in CoppeliaSim on the HRP-4 humanoid robot in scenarios of different
complexity.

Our future work will explore several directions, such as
\begin{enumerate}
    \item providing a formal proof of asymptotic optimality of the proposed
        footstep planner;
    \item developing a more general version of the proposed approach to deal
        with arbitrary terrains, removing the world of stairs assumption;
    \item implementing the presented architectures on a real humanoid robot;
    \item extending them to the case of large-scale and multi-floor environments.
\end{enumerate}

\medskip

In Chapter \ref{ch:FAPA}, we presented Feasibility-Aware Plan Adaptation (FAPA),
a module for adapting footstep plans (positions, orientations and timings) in
such a way to enhance the IS-MPC scheme of Chapter \ref{ch:ISMPC}. To do so,
we exploited the feasibility region of IS-MPC (Sect.
\ref{sec:ISMPC:feasibility-region}), using a gait feasibility constraint.
We considered two versions of the scheme:
Fixed patches FAPA (F-FAPA), where the regions assignment for placing the footsteps is 
fixed, and Variable patches FAPA (V-FAPA), where the regions can be selected
automatically. In F-FAPA, the optimization problem is formulated as a Nonlinear
Programming Problem (NLP), while in V-FAPA it is formulated as a mixed-integer 
NLP. We validated FAPA in MATLAB simulations, showing that the plan is adapted in a very
flexible way in reaction to strong pushes. In our MATLAB prototype,
the performance is fully compatible with real time in the case of F-FAPA,
while not yet in the case of V-FAPA.

Future works will be aimed at
\begin{enumerate}
    \item reimplementing FAPA in C++ in order to meet real-time requirements;
    \item introducing the on-line footstep planner of Chapter
        \ref{ch:humanoid-motion-generation-WoS}
        inside the architecture so that global replanning is possile;
    \item explore convex relaxation \cite{Marcucci2024ShortestPathsinGraphofConvexSets}
        and use a MIQP solver such as Gurobi \cite{GurobiOptimizerReferenceManual}
        to further speed up the computation;
    \item deploy FAPA on a real humanoid robot.
\end{enumerate}

\medskip

In Chapter \ref{ch:nmpc-swmr}, we presented a framework for trajectory tracking
for steerable wheeled mobile robots, which makes use of a Nonlinear MPC based
on real-time iteration. Our scheme is capable of tracking trajectories without
violating wheels' velocity constraints, while taking into account kinematic
model singularities. We have validated our approach on multiple trajectories
using the Neobotix MPO-700, showing that our scheme is always able to track
them. To the best of our knowledge, this is the first time a NMPC has been
implemented on a SWMR.

In our future works, we plan to extend the framework in several ways:
\begin{enumerate}
    \item extend the NMPC to a dual-arm mobile manipulators such as BAZAR robot
        \cite{Cherubini2019ACR}, making it interact with the environments with
        the arms while moving;
    \item implement a motion planning algorithm such as kinodynamic RRT*
        \cite{Webb2013KinodynamicRRTstar}, making the robot able to navigate
        autonomously in an environment with obstacles;
    \item further improve the performance of the framework by implementing it
        in C++ (while the scheme runs in real-time thanks to acados which
        compiles the NMPC, most of the time is taken by the auxiliary
        trajectory generation module, which completely relies on Python).
\end{enumerate}
